\documentclass[oneside, final, 14pt]{extreport}
\usepackage[utf8]{inputenc}
\usepackage[russian]{babel}
\usepackage{vmargin}
\setpapersize{A4}
\setmarginsrb{2cm}{1.5cm}{1cm}{1.5cm}{0pt}{0mm}{0pt}{13mm}
\usepackage{indentfirst}
\sloppy
\begin{document}
Лекция 07.02.2012
\section{Параграф 3}
\bfseries Обращение матриц методом Гаусса-Жордана.
\\
\normalfont
Важный вопрос.
\\
Алгебраическая постановка задачи.
\\
Пусть дана квадратная невырожденная матрица \(A (m,m), \mid A \mid \neq 0\)
\\
По определению матрица A называется обратной, если \(A^{-1}A=AA^{-1}=E\), где \(E\)-единичная матрица.
\\
Существует 2 метода обращения:
\\
1) Через алгебраические дополнение. Это неэкономичный метод \(\Rightarrow (m-1)!\) действий. А мы решаем задачи большого порядка. Например бывают трехмерные нестационарные задачи для уравнения теплопроводности порядка \(10^6\).
\\
Пусть \(A^{-1}=X\). В матрице \( X \; m^2\) элементов. Тогда имеем \begin{equation}AX=E\end{equation}
СЛАУ \( \rightarrow \Sigma_{l=1}^{m}a_{i,l}x_{l,j}=\sigma_{i,j} \) по координатам. Самый экономичный. Докажем, что можно затратить \(m^3\) операций нашим супер агоритмом.
\\
Система \(AX=f\) может распасться на \(m\) систем. Таким образом мы сократим количество действий.
\\
Обращение Гаусса-Жордана. 
\\
Введем вектор столбец \(X_j=(x_{1,j},\ldots ,x_{m,j})^T\)
\[ \sigma^{j}=(0, 0, \ldots,  0, 1, 0, \ldots, 0, 0)^T \]
Для обращения необходимо решить \(m\) систем 
\begin{equation} AX^j= \sigma^j, \; j=1, 2, \ldots , m \end{equation} 
Применим факторизацию. Пусть угловые миноры отличны от нуля \(\Rightarrow BCX^{(j)}=\sigma^{(j)}, \; CX^{(j)}=Y^{(j)}\)
\begin{equation} BY^{(j)}=\sigma^{(j)}\end{equation}
\begin{equation} CX^{(j)}=Y^{(j)}, \; j=1, 2, \ldots ,m \end{equation}
В совокупности \( m^2\) действий. Т.к.  m систем \(\Rightarrow m^3\) действий.
\\
Один раз проведем факторизацию \( \Rightarrow m^3+\frac{m^3-m}{3}\) Итого \( \frac{4}{3}m^3-\frac{m}{3}\)
\\
Если воспользоваться спецификой вида матриц, то возможно получить \(m^3\) действий. Покажем это.
\\
B -- нижняя треугольная матрица. Посмотрим решение системы (3):
\[b_{1,1}y_1^{(j)}=0 \Rightarrow y_1^{(j)}=0\]
\[b_{2,1}y_1^{(j)}+b_{2,2}y_2^{(j)}=0 \Rightarrow y_2^{j}=0\]
аналогично до \((j-1)\)
\[b_{j-1,1}y_{1}^{(j)}+b_{j-1,2}y_2^{(j)}+\cdots+b_{(j-1), i-1}y_{j-1}^{(j)}=0 \Rightarrow y_{j}^{(j)}=0,\; i\leq j-1\]
\\
\[b_{j,j}y_{j}^{(j)}=1 \Rightarrow y_j^{(j)}=\frac{1}{b_{j,j}}, \; i=j\]
\\
\[b_{i,j}y_{j}^{(j)}+b_{i,j+1}y_{j+1}^{(j)}+\cdots+b_{i,i}y_{i}^{(j)}=0 \Rightarrow
y_{i}^{(j)}=-\frac{\Sigma_{l=j}^{i-1}b_{i,l}y_{l}^{(j)}}{b_{i,i}}, \; i=j+1, j+2, \ldots, m \]
\\
Сначала фиксируем i и j и считаем количество действий. Итого 1 деление и (i-j) умножений.
\\
Отпускаем индекс i:
\[ m-j+(m-j-1)+\ldots+2+1 = \frac{m-j+1+m-j}{2} \]
умножений при фиксированном i. Еще одно деление при \((i=j)\) и \( m-j\) делений.
Итого при фиксированном   \( j \; \frac{(m-j+1)(m-j+2)}{2} \)  действий.
\\
Отпускаем индекс j, т.к. \( j=1, 2, \ldots, m \Rightarrow \Sigma_{j=1}^m \frac{(m-j+1)(m-j+2)}{2} \) действий.
\\
\bfseries Задача2: \normalfont  Доказать, что для решения системы (3) необходимо \(\frac{m(m+1)(m+2)}{6} \) и для решения системы (4):\(\frac{m(m-1)}{2}, j=1, 2,\ldots ,m \Rightarrow\) всего \(\frac{m^2(m-1)}{2}\) умножений и делений.
\\
Факторизация:  \(\frac{m^3-m}{3}+\frac{m(m+1)(m+2)}{6}+\frac{m^3-m^2}{6}=\frac{2m^3-2m+m^3+3m^2+2m+3m-3ь}{6}=m^3 \) действий на обращение матрицы
\end{document}