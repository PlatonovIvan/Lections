\documentclass[oneside, final, 14pt]{extreport}
\usepackage[utf8]{inputenc}
\usepackage[russian]{babel}
\usepackage{vmargin}
\usepackage{amsmath}
\setpapersize{A4}
\setmarginsrb{2cm}{1.5cm}{1cm}{1.5cm}{0pt}{0mm}{0pt}{13mm}
\usepackage{indentfirst}
\sloppy
\begin{document}
Лекция 07.02.2012
\\
На прошлой лекции мы получили формулу при \( b_{i,i} \neq 0 \)
\[ c_{i,j} = \begin{cases} a_{i,j}-\Sigma_{l=1}^{j-1}b_{i,l}c_{l,j},  i \geq j \\
\frac{a_{i,j}-\Sigma_{l=1}^{i-1}b_{i,l}c_{l,j}}{b_{i,i}},  i \leq j \end{cases}  \] 
\\			
Разложение возможно при определенных условиях.
\\
Сформулируем достаточные условия разложения \begin{math}A=BC\end{math}.
Утверждение: Пусть все главные угловые миноры матрицы A отличны от нуля.
\[ \triangle_1 = a_{11} \neq 0 \]
\[ \triangle_2 = \begin{bmatrix} a_{1,1} & a_{1,2} \\ a_{2,1} & a_{2,2} \end{bmatrix} \neq 0 \]
\[ \triangle_n = \begin{bmatrix}a_{1,1} & a_{1,2} & \ldots & a_{1,n} \\ a_{2,1} & a_{2,2} & \ldots & a_{2,n} \\ \ldots & \ldots  & \ldots & \ldots  \\ a_{n,1} & a_{n,2} & \ldots &a_{n,n}   \end{bmatrix}\neq 0 \; , i=1, 2, \ldots, n \] 
Тогда факторизация матрицы A возможна и находится единственным образом.
Доказательство: для удобства введем \[ \triangle_0 = 1 \]
Распишем матрицу \( A_i=B_iC_i \) как произведение. Определитель произведения матриц есть произведение определителей матриц. Тогда 
\[ \triangle_i = \mid A_i \mid = \mid B_i \mid \mid C_i \mid, \]  где \( \mid C_i \mid =1\). 
Тогда 
\[ \triangle_i=b_{11}b_{22} \cdots b_{i-1,i-1}b_{i,i} \Rightarrow b_{i,i} = \frac{\triangle_i}{\triangle_{i-1}}, \; i=1,2, \ldots, n\]
Что и требовалось доказать.
\\
Все элементы \(b_{i,i}\) отличны от нуля. 
\\
Замечание: Условие отличия от нуля всех угловых миноров является достаточным. На практике это не является жестким требованием. Например в физике и химии присутствуют самосопряженные операторы. Мы не ставили перед собой задачу доказывать утверждение с минимальными требованиями. 
\\
Для чего нужна факторизация? Ответ: для удобства решения алгебраических уравнений. 
\begin{equation} Ax=f, \; \mid A \mid \neq 0, \; A=(m,m)\end{equation} 
\\
Рассмотрим связь Метода Гаусса с разложением матрицы \( A \) на множители \( A=BC \)
Затем покажем эффективность метода.
\\
\(BCx=f \Rightarrow \)
\begin{equation}BY=f\end{equation}
\begin{equation}CX=Y\end{equation}
Решение системы (1) распалось на два. Из (2) находим \(Y\) и подставляем в (3), откуда находим \(X\)
\\
Задача1: Доказать, что нахождение матриц \( B \) и \( C\) требует \(\frac{m^3-m}{3}\) умножений и делений.
Ясно, что основная работа идет на факторизацию. Сопоставим это с методом Гаусса.
\\
Систему \(A\) сводим к верхней треугольной матрице (прямой ход):
\\
A \(\begin{bmatrix} 1 & c_{1,2} & c_{1,3} & \ldots & c_{i,j} \\ 0 & 1 & c_{2,2} & \ldots & c_{2,j} \\ \ldots & \ldots & \ldots & \ldots & \ldots \\ 0 & 0 & 0 & \ldots & 1 \end{bmatrix} \)
\\
1-я связь: распишем системы (2) и (3) по координатам
\\
\[(2): \; b_{i,1}y_1+b_{i,2}y_2+\cdots+b_{i,i}y_i=f_i \; i=1, 2, \ldots, n\]
\[(3):\; x_i+c_{i,i+1}x_{i+1}+\cdots+c_{i,m}x_m=y_i \; i=1, 2, \ldots, m\]
\\
Выразим из (2) вектор \(y\): cчитая, что \(b{i,i}\neq 0\)
\[ y_i=\frac{f_i-\Sigma_{l=1}^{i-1}b_{i,l}y_l}{b_{i,i}} \; i=1,2, \cdots, m\]
\\
Из (3) выразим вектор \(x\):
\[x_i=y_i-\Sigma_{l=i+1}^mc_{i,l}x_l\]
Посчитаем количество действий. В \(y_i\) находится \(i-1\) умножений \(+1\) деление. Итого \(=i\) действий.
\\
Т.к. \(i=1,2, \ldots ,m \Rightarrow \Sigma_{i=1}^mi=1+2+ \cdots +m=\){арифметическая прогрессия}\(=\frac{m(m-1)}{2}\) 
\\ 
В методе Гаусса на преобразование правой части (в прямом ходе метода Гаусса) уходит \(\frac{m(m-1)}{2}\) действий.
\\
Рассмотрим (3):
\\
\(x_i=y_i-\Sigma c_{i,l}x_l \rightarrow(m-i)\) умножений при фиксированном \(i\). Теперь отпускаем \(i \Rightarrow (m-1)+(m-2)+\cdots +1=\frac{m(m-1)}{2}\). Столько же в обратном ходе метода Гаусса. Таким образом метод Гаусса требует \(\frac{m^3}{3}+m^2- \frac{m}{3}\). Большая часть операций уходит на факторизацию.
\\
   



\end{document}
